%%% 1206.tex

%% Copyright (C) 2012 LRDE.

%% Permission is granted to copy, distribute and/or modify this document
%% under the terms of the GNU Free Documentation License, Version 1.2
%% or any later version published by the Free Software Foundation;
%% with the Invariant Sections being just ``Copying this document'',
%% no Front-Cover Texts, and no Back-Cover Texts.

%% A copy of the license is provided in the file COPYING.DOC.

\documentclass{techrep} % You can pass the french option if you like.
\usepackage[T1]{fontenc}
\usepackage{amsmath, bm}
\usepackage{fancyhdr}
\usepackage{array}
\usepackage{stmaryrd}
\usepackage{graphicx}
\usepackage{gensymb}
\usepackage{vaucanson-g}
\usepackage{amsfonts}
\usepackage{float}
\usepackage{verbatim}
\usepackage{makeidx}
\usepackage{lmodern}
\usepackage{amsmath}
\usepackage{amsthm}
\usepackage{amsfonts}
\usepackage{tikz}
\usepackage{listings}
\usetikzlibrary{automata,positioning}
\title{SpeakerID - Spherical Discriminant Analysis}
\author{Victor Lenoir} \revision$LastChangedRevision: 2340 $
\date{January 2013} \email{lenoir@lrde.epita.fr}
%% \www{URL}{TEXT}

\definecolor{dkgreen}{rgb}{0,0.6,0}
\definecolor{gray}{rgb}{0.5,0.5,0.5}
\definecolor{mauve}{rgb}{0.58,0,0.82}

\lstset{ %
  language=Python,                % the language of the code
  basicstyle=\footnotesize,           % the size of the fonts that are used for the code
  numbers=left,                   % where to put the line-numbers
  numberstyle=\tiny\color{gray},  % the style that is used for the line-numbers
  stepnumber=2,                   % the step between two line-numbers. If it's 1, each line
  % will be numbered
  numbersep=5pt,                  % how far the line-numbers are from the code
  backgroundcolor=\color{white},      % choose the background color. You must add \usepackage{color}
  showspaces=false,               % show spaces adding particular underscores
  showstringspaces=false,         % underline spaces within strings
  showtabs=false,                 % show tabs within strings adding particular underscores
  frame=single,                   % adds a frame around the code
  rulecolor=\color{black},        % if not set, the frame-color may be changed on line-breaks within not-black text (e.g. commens (green here))
  tabsize=2,                      % sets default tabsize to 2 spaces
  captionpos=b,                   % sets the caption-position to bottom
  breaklines=true,                % sets automatic line breaking
  breakatwhitespace=false,        % sets if automatic breaks should only happen at whitespace
  title=\lstname,                   % show the filename of files included with \lstinputlisting;
  % also try caption instead of title
  keywordstyle=\color{blue},          % keyword style
  commentstyle=\color{dkgreen},       % comment style
  stringstyle=\color{mauve},         % string literal style
  escapeinside={\%*}{*)},            % if you want to add a comment within your code
  morekeywords={*,...}               % if you want to add more keywords to the set
}
%10 lines


\summary{ The purpose of a speaker verification system is to check
  whether a hypothesized speaker is really the author of a speech
  utterance. Currently, the best performances are performed using a
  mapping of each speaker utterance to a low dimensional vector called
  I-vector. The speaker verification score is computed thanks to a
  cosine distance between these two vectors representing both a
  speaker.


  In this paper, we describe a dimensionality reduction technique
  called Spherical Discriminant Analysis (SDA).  The goals of this
  projection are to maximize the cosine distance between different
  speaker's uterrances and minimize the cosine distance between same
  speaker's utterances; it has been shown that SDA subspace, which is
  more suitable for the cosine distance than Linear Discriminant
  Analysis (LDA), yields superior performance in face recognition
  task. We will compare the SDA subspace performance with the standard
  LDA approach.}

\frenchsummary{La r\^ole de la v\'erification de locuteur est de
  v\'erifier l'identit\'e pr\'esum\'ee d'un segment de parole. Actuellement,
  les meilleures performances sont obtenus par un mapping de chaque
  segment de parole d'un locuteur vers un vecteur appel\'e I-vector. Le
  score de la v\'erification de locuteur est calcul\'e par une distance
  cosine entre ces deux vecteurs repr\'esentant chacun un locuteur.

  Ce rapport d\'ecrit une technique de r\'eduction de dimension appel\'ee
  Spherical Discriminant Analysis (SDA). Les objectifs de cette
  projection sont de maximiser la distance cosinus entre deux locuteurs
  diff\'erents et de minimiser la distance cosinus entre deux m\^eme
  locuteurs; il a \'et\'e montr\'e que le sous-espace de la SDA, qui est plus
  appropri\'e pour la distance cosinus que la Linear Discriminant Analysis
  (LDA), obtient de meilleur performance en reconnaissance faciale. Nous
  allons comparer les performances obtenues par la SDA avec celles obtenues par la LDA.}

\keywords{speaker verification, dimensionality reduction, subspaces,
  ivector, cosine distance}

\begin{document}

\section*{Copying this document}
Copyright \copyright{} 2012 LRDE.

Permission is granted to copy, distribute and/or modify this document under
the terms of the GNU Free Documentation License, Version 1.2 or any later
version published by the Free Software Foundation; with the Invariant Sections
being just ``Copying this document'', no Front-Cover Texts, and no Back-Cover
Texts.

A copy of the license is provided in the file COPYING.DOC.

\tableofcontents

\newpage
%TODO: remerciements
\chapter*{Introduction}
%Une introduction avec le contexte général de votre travail, une
%description des problèmes auxquels vous voulez répondre et comment
%vous y répondez.  L’introduction doit capturer le lecteur et lui
%donner envie de lire la suite. Elle doit donner assez d’informations
%pour que celui-ci puisse comprendre l’intérêt de votre travail et la
%portée de vos résultats.  Plan
\chapter{Context: Speaker verification system}
%Présenter plus en détail le cadre de travail, la problématique, et
%l’existant.

%S’il y a des prérequis, des notions que devraient
%connaître le lecteur, c'est ici.

%Par exemple le
%projet Vaucanson a souvent besoin de revenir sur les bases des
%transducteurs; il est alors stupide de taper à nouveau cette
%introduction. Il vaut mieux maintenir une base commune, et clairement
%faire apparaître ses auteurs.


Nowadays, in a lot of secure applications, people have to prove their own identity to go into
a critical system. Most of these applications are based on password authentications as in bank
cash points or in the access of some protected buildings. However in such systems, a robber can
easily find the password and use it to gain access to the critical area. For this reason, more and
more authentication systems are based on biometrical features like fingerprints, iris or voice
which present more inviolable features. Indeed, the study of these biometrical features is an
expanding research field. Furthermore, it is known that fingerprints have been used extensively
in criminal investigations for a long time. Today, voice recognition systems are beginning to
have a legal status in some countries as a proof to authenticate a speaker on a tape recording.
In this report, we consider the problem of voice authentication, generally called the speaker
verification problem. More precisely, we focus on the text-independent speaker verification i.e
we do not consider the uttered text.\\\\
In our speaker verification task, speakers are first modelled from enrolment data coming
from phone recordings. During the verification task, these models enable to check whether
a segment of speech is uttered by a hypothesised speaker. For a few years, the community
has been more and more interested in resolving this verification problem even if the enrolment
conditions are different during the training step and the testing step.

\section{Speaker verification system}

\tikzstyle{cont}=[shape=rectangle,minimum
  size=1.0cm,draw=blue!70,fill=blue!30,font=\small]

\tikzstyle{contred}=[shape=rectangle,minimum
  size=1.0cm,draw=red!70,fill=red!30,font=\small]

\begin{center}
  \begin{tikzpicture}[scale=0.8,font=\scriptsize]
    \node[cont,initial above] (s1) at (0,0) {Audio signal};
    \node[cont] (s2) at (0,-2) {Audio features}
    edge [<-] node[left] {Features extraction} (s1);
    \node[cont] (s3) at (0,-4) {Speaker features}
    edge [<-] node[left] {Voice Activity Detection/Speaker diarization} (s2);
    \node[cont] (s4) at (0,-6) {Speaker model}
    edge [<-] node[left] {Modeling} (s3);
    \node[cont] (s5) at (0,-8) {Low-dimensional speaker model}
    edge [<-] node[auto] {\textcolor{red}{Dimension reduction}} (s4);
    \node[cont] (s6) at (5,-10) {Low-dimensional speaker model};
    \node[cont] (s7) at (0,-10) {Score}
    edge [<-] node[auto] {Scoring} (s5)
    edge [<-] node[auto] {Scoring} (s6);
    \node[cont] (s8) at (0,-12) {Decision}
    edge [<-] node[auto] {Threshold} (s7);
  \end{tikzpicture}
\end{center}

In our system:
\begin{description}
\item[Features] Cepstral vectors (accoustic parameters);
\item[Speaker model] Identity-Vector (Ivector, 600-dim vector which
  represents the speaker);
\item[Low-dimensional speaker model] 400-dim vector obtained by
  Probabilistic Linear Discriminant Analysis (PLDA)
\item[Score] Cosine-Distance between two low-dimensional speaker models;
\item[Decision] Whether the speakers are the same or not.
\end{description}

\section{Dimension reduction}

\chapter{Spherical Discriminant Analysis}
\section{Principle}
%ce qu'on veut faire et pourquoi on veut le faire
%comment on en est arrivé à l'algo
\section{Algorithm}
%algo here

%Corps du travail
\chapter{Experiments and results}
\section{Experiments}
\section{Results}
\section{Discuss}
% Previous work, Related work, future work
\chapter*{Conclusion}
%Quelques paragraphes. Une conclusion qui résume clairement le contenu de votre rédaction, dans le même ordre que le contenu présenté était détaillé. La conclusion doit ensuite mettre l’accent sur les principaux intérêts / fonctionnalités de ce que vous avez présenté et mettre en avant votre travail.
%Ne pas se contenter de faire une resucée du résumé. D’ailleurs, ce n’est pas une «synthèse», c’est une «conclusion». En d’autres termes, ne pas s’arrêter à un résumé de l’article, mais finir en ouvrant le débat : quelles pistes restent encore à explorer, quels nouvelles questions ou problèmes se posent, quels autres sujets semblent liés etc.

\bibliography{final_report} \nocite{*}
\end{document}

%%% 1206.tex ends here.
